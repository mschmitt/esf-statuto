\documentclass[11pt]{article}
\setlength{\parindent}{0pt}
\usepackage[utf8]{inputenc}
\usepackage[T1]{fontenc}
\usepackage{underscore}
\usepackage{german}
\usepackage{geometry}
\usepackage{enumitem}
\usepackage{helvet}
\usepackage{url}
\urlstyle{same}
\geometry{a4paper,left=25mm,right=20mm,top=20mm,bottom=20mm}
\renewcommand{\familydefault}{\sfdefault}
\renewcommand\thesection{§ \arabic{section}}

\begin{document}

\begin{titlepage}
\vspace*{\fill}
\centering
\huge 
SATZUNG

der

ESPERANTO-GESELLSCHAFT

Frankfurt am Main e.V.

Gegründet 1904

\vspace{1cm}
\normalsize
Text der Satzung vom 6.2.1992

Abschrift vom 15.4.2020
\vspace*{\fill}
\end{titlepage}
\newpage

\section{Name und Sitz des Vereins}
\begin{enumerate}[label=\arabic*)]
	\item Der Verein führt den Namen "`Esperanto-Gesellschaft Frankfurt am Main"', abgekürzt EGF. In Esperanto lautet der Name "`Esperanto-Societo Frankfurt"', abgekürzt ESF.
	\item Sitz des Vereins ist Frankfurt am Main.
	\item Der Verein besteht seit dem 30. September 1904 und ist seit dem 5.2.1907 in das Vereinsregister eingetragen.
\end{enumerate}

\section{Ziele und Zwecke}
\begin{enumerate}[label=\arabic*)]
	\item Der Verein will die internationale Sprache Esperanto für eine leichtere Verständigung zwischen den Völkern pflegen und zum allgemeinen Nutzen verbreiten.
	\item Der Verein verfolgt ausschließlich und unmittelbar gemeinnützige Zwecke im Sinne des Abschnitts "`Steuerbegünstigte Zwecke"' der Abgabenordnung von 1977. Er ist selbstlos tätig und verfolgt nicht in erster Linie eigenwirtschaftliche Zwecke.
	\item Mittel des Vereins dürfen nur für satzungsgemäße Zwecke verwendet werden. Die Mitglieder erhalten keine Zuwendungen aus den Mitteln des Vereins. Es darf keine Person durch Ausgaben, die dem Zweck der Körperschaft fremd sind oder durch unverhältnismäßig hohe Vergütungen begünstigt werden.
\end{enumerate}

\section{Mittel}
\begin{enumerate}[label=\arabic*)]
	\item Um die Satzungsziele zu erreichen, fördert der Verein Kontakte zwischen den Freunden der internationalen Sprache Esperanto in aller Welt mit Esperantisten aus Frankfurt um Umgebung.
	\item Er organisiert Veranstaltungen zur praktischen Anwendung der Sprache, informiert die Öffentlichkeit über Esperanto und führt bei Bedarf Sprachkurse durch.
	\item Der Verein strebt die Zusammenarbeit mit anderen Organisationen an, die sich der Förderung von Esperanto oder der internationalen Verständigung widmen. Insbesondere schließt er sich dem Deutschen Esperanto-Bund e.V. als Untervereinigung an.
	\item Der Verein unterhält eine Bibliothek.
\end{enumerate}

\section{Geschäftssprachen und Geschäftsjahr}
\begin{enumerate}[label=\arabic*)]
	\item Geschäftssprachen sind Deutsch und Esperanto.
 \item Geschäftsjahr ist das Kalenderjahr.
\end{enumerate}

\section{Arten der Mitgliedschaft}
\begin{enumerate}[label=\arabic*)]
	\item Mitglied kann jede natürliche oder juristische Person werden, die bereit ist, die Ziele des Vereins zu unterstützen und die Satzung anzuerkennen.
	\item Der Verein besteht aus ordentlichen und fördernden Mitgliedern sowie aus Ehrenmitgliedern.
	\item Ordentliche Mitglieder können nur natürliche und juristische Personen werden, die den Verein durch besondere finanzielle Zuwendungen unterstützen.
	\item Fördernde Mitglieder können natürliche und juristische Personen werden, die den Verein durch besondere finanzielle Zuwendungen unterstützen.
	\item Ehrenmitglieder können Persönlichkeiten werden, die sich um die Belange des Vereins oder um Esperanto besonders verdient gemacht haben.
	\item Ordentliche Mitglieder der EGF sind in der Regel gleichzeitig Mitglieder im Deutschen Esperanto-Bund e.V., es sei denn, sie lehnen dies ausdrücklich ab. In diesem Fall werden sie als örtliche Mitglieder geführt.
\end{enumerate}

\section{Erwerb der Mitgliedschaft}
\begin{enumerate}[label=\arabic*)]
	\item Der Erwerb der ordentlichen oder der fördernden Mitgliedschaft bedarf einer schriftlichen Beitrittserklärung. Der Vorstand entscheidet über die Aufnahme. Lehnt er sie ab, hat der Beitrittswillige das Recht zur Beschwerde bei der nächsten Mitgliederversammlung, die endgültig entscheidet.
	\item Ehrenmitglieder werden auf Vorschlag des Vorstandes von der Mitgliederversammlung ernannt.
\end{enumerate}

\section{Rechte und Pflichten der Mitglieder}
\begin{enumerate}[label=\arabic*)]
	\item Ordentliche Mitglieder und Ehrenmitglieder sind berechtigt,
	\begin{enumerate}[label=\alph*)]
		\item an den Vereinsveranstaltungen teilzunehmen
		\item die Vereinseinrichtungen zu benutzen
		\item die Vereinsmitteilungen kostenlos zu beziehen
		\item bei Mitgliederversammlungen abzustimmen
		\item zu wählen und gewählt zu werden, sofern sie volljährig sind
		\item dem Vorstand und der Mitgliederversammlung Vorschläge zu unterbreiten.
	\end{enumerate}
	\item Fördernde Mitglieder haben die unter §7,1 genannten Rechte, mit Ausnahme von d) und e).
	\item Alle Mitglieder haben die Pflicht, daran mitzuwirken, die Ziele des Vereins zu erreichen sowie Schaden von ihm abzuwenden.
	\item Sie sind insbesondere verpflichtet
		\begin{enumerate}[label=\alph*)]
		\item Satzung sowie Beschlüsse des Vorstandes und der Mitgliederversammlung zu beachten bzw. zu verwirklichen
		\item die jährlichen Mitgliedsbeiträge bis Ende Januar zu bezahlen. Dies gilt auch für die Beiträge, die der Verein an den Deutschen Esperanto-Bund weiterleitet.
		\item alle sonstigen Zahlungsverpflichtungen zu erfüllen.
		\item Adressen- und Namensänderungen dem Vorstand unverzüglich mitzuteilen.
	\end{enumerate}
	\item Ehrenmitglieder sind von der Pflicht der Beitragszahlung befreit.
	\item Für bestimmte Gruppen (z.B. Jugendliche, Familienangehörige) können besondere Rechte und Pflichten von der Mitgliederversammlung festgelegt werden.
	\item Rechte und Pflichten, die sich aus der Mitgliedschaft im Deutschen Esperanto-Bund e.V. ergeben, bleiben von diesen Bestimmungen unberührt.
\end{enumerate}

\section{Beendigung der Mitgliedschaft}
\begin{enumerate}[label=\arabic*)]
	\item Die Mitgliedschaft erlischt durch Austritt, Ausschluss, Streichung aus der Mitgliederliste oder Tod.
	\item Der Austritt ist dem Vorstand schriftlich zu erklären. Diesem Wunsch ist zu entsprechen. Er wirkt auf das Ende des laufenden Kalenderjahres.
	\item Durch den Vorstand kann ein Mitglied ausgeschlossen werden, wenn es trotz zweimaliger schriftlicher Mahnung die rückständigen Beiträge nicht bezahlt. Gegen einen Ausschluss steht dem Mitglied das Recht der Beschwerde bei der nächsten Mitgliederversammlung zu, die endgültig entscheidet. Während der Zeit der Säumnis besteht kein Anspruch auf Lieferung der Vereinsmitteilungen.
	\item Durch die Mitgliederversammlung kann ein Mitglied ausgeschlossen werden, wenn es seine Pflichten grob verletzt oder sich sonst der Mitgliedschaft als unwürdig erweist.
	\item Mitglieder, deren Verbleib unbekannt ist, können aus der Mitgliederliste gestrichen werden. Dies ist der Mitgliederversammlung mitzuteilen.
	\item Die Mitgliederversammlung kann Ehrenmitgliedern diese Eigenschaft unter den Voraussetzungen des § 8,4 wieder entziehen.
\end{enumerate}

\section{Organe des Vereins}
\begin{enumerate}[label=\arabic*)]
	\item Organe des Vereins sind
	\begin{enumerate}[label=\alph*)]
		\item die Mitgliederversammlung
		\item der Vorstand
		\item die Kassenprüfer
		\item etwaige Abwickler.
	\end{enumerate}
	\item Die Mitglieder der Vereinsorgane üben ihre Tätigkeit ehrenamtlich aus. Sie erhalten jedoch Auslagen erstattet.
\end{enumerate}

\section{Mitgliederversammlung}
\begin{enumerate}[label=\arabic*)]
	\item Die Mitgliederverammlung besteht aus Mitgliedern des Vereins.
	\item Einmal jährlich, möglichst im 1. Quartal, beruft der Vorstand eine ordentliche Mitgliederversammlung ein. Dazu sind die Mitglieder mindestens 3 Wochen vorher unter Angabe der Tagesordnung schriftlich einzulagen.
	\item Eine außerordentliche Mitgliederversammlung wird vom Vorstand auf eigenen Beschluss oder auf Verlangen von mindestens einem Viertel der stimmberechtigten Mitglieder unter Angabe eines Grundes einberufen. Termin und Tagesordenung sind den Mitgliedern mindestens eine Woche vorher schriftlich bekanntzugeben.
	\item Anträge für die ordentliche Mitgliederversammlung sind spätestens zwei Wochen vorher beim Vorsitzenden schriftlich einzureichen.
	\item Die Mitgliederversammlung ist zuständig für
	\begin{enumerate}[label=\alph*)]
		\item Wahl, Entlastung und Abberufung des Vorstandes oder einzelner Vorstandsmitglieder, der Kassenprüfer sowie etwaiger Abwickler
		\item Festsetzung der Mitgliedsbeiträge
		\item Beschluss über den Haushaltsplan
		\item Beschlussfassung über Satzung und Satzungsänderungen
		\item die Debatte und den Beschluss über Anträge, die die Tätigkeit des Vereins zur Erreichung seiner Ziele festlegen.
		\item Ernennung von Ehrenmitgliedern
		\item Auflösung des Vereins		
	\end{enumerate}
	\item Über die Mitgliederversammlung ist eine Niederschrift anzufertigen, die vom Protokollführer und dem Versammlungsleiter zu unterzeichnen ist. Sie ist der nächsten ordentlichen Mitgliederversammlung zur Genehmigung vorzulegen. Eventuelle Widersprüche sind möglichst sofort zu klären, gegebenenfalls ist ein Dringlichkeitsantrag zulässig.
	\item Wahlen und Abstimmungen finden offen statt. Es muss jedoch geheim gewählt werden, wenn mindestens ein Mitglied das verlangt.
	\item Die Mitgliederversammlung ist beschlussfähig, wenn sie ordnungsgemäß einberufen wurde und mindestens drei Mitglieder anwesend sind. Jedes anwesende Mitglied hat eine Stimme. Es entscheidet die einfache Mehrheit der abgegebenen Stimmen. Bei Stimmgleichheit gilt ein Antrag als abgelehnt.
	\item Bei Satzungsänderungen ist eine Mehrheit von zwei Dritteln, bei Auflösung des Vereins eine Mehrheit von drei Vierteln der anwesenden stimmberechtigten Mitglieder erforderlich.
	\item Wird über die Entlastung einzelner Personen abgestimmt, haben die Betreffenden kein Stimmrecht.
	\item Die Mitgliederversammlung kann sich eine Geschäftsordnung geben.
\end{enumerate}

\section{Vorstand}
\begin{enumerate}[label=\arabic*)]
	\item Der Vorstand setzt sich zusammen aus
	\begin{enumerate}[label=\alph*)]
		\item dem 1. Vorsitzenden
		\item dem 2. Vorsitzenden
		\item dem Rechnungsführer
		\item dem Schriftführer
		\item den bis zu 4 Beisitzern.
	\end{enumerate}
	\item Beide Geschlechter sollen vertreten sein.
	\item den Vorstand gemäß §26 BGB bilden der 1. Vorsitzende, der 2. Vorsitzende und der Rechnungsführer. Jeder ist einzeln vertretungsberechtigt.
	\item Die Mitglieder des Vorstands werden von der Mitgliederversammlung aus den Reihen der Mitglieder für die Dauer von 2 Jahren gewählt. Sie führen ihr Amt bis zur Neuwahl oder Wiederwahl aus.
	\item Der Vorstand ist zuständig für
	\begin{enumerate}[label=\alph*)]
		\item Führung der laufenden Geschäfte
		\item Verwirklichung der Beschlüsse der Mitgliederversammlung
		\item Vertretung des Vereins nach außen
		\item Verwaltung des Vereinsvermögen
		\item Aufstellung eines Haushaltsplans
		\item Einberufung, Vorbereitung und Durchführung der Mitgliederversammlung
	\end{enumerate}	
	\item Er gibt der Mitgliederversammlung einen jährlichen Tätigkeitsbericht.
	\item Er kann für einzelne Vorhaben Kommissionen einsetzen und ihnen in begrenztem Umfang Entscheidungsbefugnis übertragen.
	\item Scheidet ein Vorstandsmitglied vorzeitig aus, so ergänzt sich der Vorstand durch Zuwahl eines ordentlichen Mitglieds bis zur nächsten ordentlichen Mitgliederversammlung. 
	\item Vorstandssitzungen werden mindestens zweimal im Jahr vom 1. Vorsitzenden einberufen. Sie sind in der Regel nicht öffentlich. Sie müssen außerdem einberufen werden, wenn dies mindestens die Hälfte der Vorstandsmitglieder schriftlich unter Angabe eines Grundes verlangt.
	\item Bei Abstimmungen gilt § 10,8 Satz 1-4 entsprechend.
	\item Der Vorstand kann sich eine Geschäftsordnung geben.
\end{enumerate}

\section{Kassenprüfer}
\begin{enumerate}[label=\arabic*)]
	\item Die zwei Kassenprüfer haben die Aufgabe, Rechnungsführung, Zweckmäßigkeit der Ausgaben und Jahresabschluß einmal jährlich zu prüfen. Über Verlauf und Ergebnis ist dem Vorstand und der Mitgliederversammlung Bericht zu erstatten. Ist einer der beiden Kassenprüfer zum Zeitpunkt der Mitgliederversammlung verhindert, ist dessen Prüfung nachzuholen und dem Vorstand zu bestätigen.
	\item Die Kassenprüfer werden von der Mitgliederversammlung für 2 Jahre gewählt. Unmittelbare Wiederwahl ist nur einmal zulässig.
	\item Vorstandsmitglieder dürfen nicht gleichzeitig Kassenprüfer sein.
\end{enumerate}

\section{Vermögensangelegenheiten}
\begin{enumerate}[label=\arabic*)]
	\item Der Verein bringt seine Mittel aus den Mitgliedsbeiträgen, aus Spenden, aus etwaigen Entgelten für die Benutzung der Vereinseinrichtungen und aus sonstigen Zuwendungen auf. Spenden sind in den Rechnungsunterlagen besonders auszuweisen.
\end{enumerate}

\section{Abwickler}
\begin{enumerate}[label=\arabic*)]
	\item Bei der Auflösung des Vereins haben die Vorstandsmitglieder oder die von der letzten Mitgliederversammlung gewählten Abwickler die Abwicklung durchzuführen.
	\item Für die Abwickler gelten sinngemäß die Vorschriften über den Vorstand.
\end{enumerate}

\section{Verwendung des Vereinsvermögens}
\begin{enumerate}[label=\arabic*)]
	\item Das Vermögen darf in keinem Falle den Mitgliedern zufallen. Es ist bei einer Auflösung oder Aufhebung des Vereins oder bei Wegfall seines bisherigen Zwecks dem als gemeinnützig anerkannten Deutschen Esperanto-Bund e.V. oder seinem Rechtsnachfolger zuzuführen.
	\item Ist ein solcher nicht vorhanden, so ist das Vermögen an eine gemeinnützige oder öffentliche Stiftung, die der Förderung von Esperanto dient, zuzuwenden.
\end{enumerate}

\section{Inkrafttreten}

Diese Satzung tritt nach Annahme durch die Mitgliederversammlung mit der Genehmigung durch das Amtsgericht in Kraft.

Diese Satzung wurde am 5.2.1987 von der Mitgliederversammlung angenommen. Die Genehmigung durch das Amtsgericht wurde am 6.8.1987 erteilt.

Damit tritt gleichzeitig die Satzung vom 7. Februar 1980 außer Kraft.

\section*{Satzungsänderung}

Die beigefügte Satzung der Esperanto-Gesellschaft Frankfurt (Main) e.V. wurde durch Beschluß der Mitgliederversammlung vom 6.2.1992 geändert. Der Text des § 15 (nicht die Überschrift) ist zu streichen. Er lautet nun wie folgt:

"`Bei Auflösung des Vereins oder bei Wegfall steuerbegünstigter Zwecke fällt das Vermögen des Vereins an den Deutschen Esperanto-Bund e.V. oder seinem Rechtsnachfolger, der es unmittelbar und ausschließlich für gemeinnützige oder mildtätige Zwecke zu verwenden hat."'

\newpage
\begin{appendix}
Bei der manuellen Abschrift des Dokuments am 15.04.2020 wurden folgende Berichtigungen vorgenommen:
\begin{itemize}
	\item Titelseite: Punkt bei e.V ergänzt.
	\item § 1 Absatz 1): Trennstrich Frank-furt entfernt.
	\item § 2 Absatz 3): Trennstriche Körper-schaft und unver-hältnismäßig entfernt.
	\item § 4: Druckfehler zwischen Überschrift und erstem Absatz entfernt.
	\item § 7 Absatz 4): Alphabetische Nummerierung beginnend bei g) auf a) zurückgesetzt.
	\item § 7 nach Absatz 4): Absatznummerierung ging mit 2) weiter; mit 5) fortgesetzt.
	\item § 7 Absatz 6 (ehem. 3): Klammer bei "`z.B."' geschlossen. Trennstrich be-sondere entfernt.
	\item § 8 Absatz 2): Dopplung "`ist dem Vorstand ist dem Vorstand"' entfernt.
	\item § 8 Absatz 3): Trennstrich Vereins-mitteilungen entfernt.
	\item § 10 Absatz 5): Alphabetische Nummerierung beginnend bei e) auf a) zurückgesetzt.
	\item § 10: Schließende Klammer hinter der Nummerierung entfernt.
	\item § 10 Absatz 1) Buchstabe e): Nummerierungszeichen korrigiert.
	\item § 10 Absatz 1): Rechnungsprüfer ersetzt durch Rechnungsführer.
	\item § 10 Absatz 5): Vereinsvermügen zu Vereinsvermögen korrigiert.
	\item § 10 Absatz 9): Trennstrich Vorsit-zendenden entfernt. "`Vorsitzendenden"' korrigiert. Trennstrich Vor-standsmitglieder entfernt.
	\item § 12 Absatz 1): "`Kassenprüfer haben die Aufgabe, die Kassenprüfung zu prüfen"' - Ersetzt durch Rechnungsführung.
	\item § 12 Absatz 1): Trennstriche Be-richt und Mit-gliederversammlung entfernt.
	\item § 14 Absatz 1): "`Vostandsmitglieder"' korrigiert.
	\item § 16 Absatz 2): "`zuwenden"' korrigiert.
\end{itemize}

Weitere Anmerkungen zur Abschrift:
\begin{itemize}
	\item Die Abschrift des Dokuments wurde mit Texmaker erstellt und kann mit TeX Live oder MiKTeX gesetzt werden:
	\begin{itemize}
		\item \url{https://www.xm1math.net/texmaker/}
		\item \url{https://www.tug.org/texlive/}
		\item \url{https://miktex.org/}
	\end{itemize}
	\item Die Quelldatei "`Satzung_1992_Abschrift_2020.tex"' ist unter der folgenden Adresse zugänglich: \url{https://github.com/mschmitt/esf-statuto}
	\item Die Rechtschreibung wurde nicht an die zum Zeitpunkt der Abschrift geltende Regelung angepasst.
	\item § 9 Absatz 1 Buchstabe c): Kassenprüfer ist an dieser Stelle ausdrücklich gemeint, wie im folgenden in § 12 beschrieben.
\end{itemize}

\end{appendix}
\end{document}
